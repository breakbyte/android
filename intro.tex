\section{Introduction}
	In the era of data explosion, people may be helpless in the face of huge data, because it is difficult for people to find items they are interested in from thousands of different candidates. Recommender system was born to solve this problem. Recommender system can use the user's historical behavior and other information to get the recommendation list that the user may be interested in through the recommendation algorithm, so as to recommend the appropriate items to the user. Nowadays, this recommender system has been widely deployed in various network services, and has brought great economic benefits.
	
	In order to make the recommender system have more personalized recommendation and higher performance, researchers continue to put forward a variety of new frameworks and methods. Traditional recommendation algorithms include neighborhood-based, graph-based, ranking-based, association-rule-based, matrix-factorization-based and so on. Recommender system based on deep learning makes use of the advantages of deep learning model, which opens a new idea for the development of recommender system.
	
	Meanwhile, due to the wide application of recommender system, attacks against recommender system are also emerging one after another. Research shows that recommender systems are vulnerable to data poisoning attacks. Data poisoning attack mainly takes advantage of the characteristics that recommender system needs to collect  information of many users. By injecting a large number of carefully crafted fake user information into the recommender system, attacker can control the recommender system to recommend the target item. For example, the attacker attacks the recommender system deployed on the e-commerce system, so as to recommend the specified goods to the majority of normal users, so as to obtain improper benefits. Although the recommender system based on deep learning has received extensive attention and has been put into use, the research on data poisoning attack of recommender system based on deep learning is still relatively few, and the hit rate of the attack needs to be improved.
	
	In this work, we propose a new data poisoning attack method for recommender system based on deep learning. The attacker's goal is to promote the selected target items as much as possible in the recommender system of deep learning. Therefore, the attacker needs to inject a certain number of fake users into the recommender system. Each fake user needs to rate the target items and other non-target items. The key point is that we need to select appropriate rated items for each fake user to improve the possibility of recommendation of the target item and avoid possible detection.
	
	The difficulty in solving this problem lies in:xxxx
	
	For this, our solution is:xxxx
	
	We evaluate our data poisoning attack. (summarize the experimental results)xxxx
	
	The contributions of our paper are summarized as follows: 
	\begin{itemize}
		\item We propose a new method of data poisoning attack on recommender system based on deep learning.
		\item We xxxx(describe the solution)
		\item We evaluate our attack and compare it with existing attacks on real-world datasets.
	\end{itemize}
	
	
	
	
	
